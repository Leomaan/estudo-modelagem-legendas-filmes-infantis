\documentclass[twocolumn, 9pt]{article}

\usepackage[utf8]{inputenc}
\usepackage[T1]{fontenc}
\usepackage[brazil]{babel}


\usepackage[left=48pt, right=42pt, top=48pt, bottom=60pt]{geometry}


\usepackage{mathpazo}
\usepackage[scaled]{helvet}


\usepackage[alf, abnt-etal-cite=4, abnt-year-extra=plain]{abntex2cite}

\usepackage{xcolor}
\usepackage[explicit]{titlesec}
\usepackage{lineno}



\definecolor{DarkBlue}{RGB}{0,48,58}

\titleformat{\section}
  {\large\sffamily\bfseries\color{DarkBlue}}
  {\thesection.}
  {0.5em}
  {#1}
  []


\usepackage{graphicx}


\begin{document}


\twocolumn[
  \begin{center} 
    {\sffamily\bfseries\huge Estudo e modelagem para legendas de filmes infantis\par}
    \vspace{1.5em}
    {\large Leoman Cássio Almeida dos Santos\par}
    \vspace{1em}
    {\normalsize Universidade do Estado da Bahia\par}
    \vspace{2em}
  \end{center}


  \noindent\textbf{Abstract.} 
  Este estudo apresenta uma análise computacional em larga escala de um corpus de legendas de filmes infantis, 
  abrangendo 4 décadas. Utilizando técnicas de Processamento de Linguagem Natural (PLN), o trabalho foca em três eixos
   principais: (1) a transmissão de valores sociais ao decorrer das décadas; (2) a evolução linguística de termos e (3) a representação e variação das emoções nos diálogos. A metodologia baseia-se em pré-processamento textual, aplicação de dicionários lexicais e análise quantitativa de frequências, complementada por visualizações temporais.Os resultados apontam para a década de 2000 como um período de pico na frequência de termos associados a emoções, valores sociais como "Família" e "Amizade", e elementos de fantasia, seguido por uma normalização na década de 2010. Notavelmente, a proporção entre as emoções permaneceu estável ao longo do tempo, sugerindo que os filmes se tornaram mais expressivos em volume, mas não em sua estrutura emocional.
  \par\vspace{1em} 

  \noindent\textbf{Keywords:} Processamento de Linguagem Natural; Legendas; Processamento computacional
  \par\vspace{3em} 
]


\section{Introdução}

O cinema infantil ultrapassa o mero entretenimento para se estabelecer como um dos mais poderosos agentes de socialização da cultura contemporânea. Dada sua atração e influência sobre as crianças, o cinema é um objeto de estudo fundamental para compreender as mensagens transmitidas à infância \cite{Borges2023Filmes}.

Apesar disso, grande parte das análises sobre o tema se concentra em estudos de caso qualitativos, que, embora aprofundados, não conseguem capturar as tendências gerais que emergem ao longo de décadas.
Diante desse cenário, o Processamento de Linguagem Natural (PLN) surge como a ferramenta ideal, permitindo a análise sistemática de grandes volumes de texto, como as legendas de filmes — um material que, por sua proximidade com a linguagem oral, é um excelente recurso para estudos linguísticos.\cite{inproceedings8b3efc2c}

Este trabalho aplica exatamente essa abordagem em um corpus inédito de 433 legendas (1980-2016) para contribuir no entendimento das transformações do cinema infantil. Nossa análise investiga três aspectos centrais: a evolução dos valores sociais, o mapeamento do vocabulário infantil e a variação na intensidade emocional dos diálogos.
\section{Metodologia }
Este estudo adotou uma abordagem quantitativa baseada em Processamento de Linguagem Natural (PLN) para analisar sistematicamente um corpus de legendas de filmes infantis. A metodologia seguiu um pipeline rigoroso em cinco etapas: aquisição do corpus, pré-processamento inicial, conversão e limpeza textual, agrupamento temporal e análise computacional.

\subsection{Coleta e Construção do Corpus}
O corpus foi constituído mediante um processo rigoroso de coleta e preparação de dados textuais. Inicialmente, realizaram-se o download e a descompactação de 433 arquivos de legendas em formato ZIP da plataforma Opensubtitles. Procedeu-se à conversão integral para o formato SubRip (SRT), com padronização de nomenclatura incluindo título do filme e ano de produção. Na sequência, executou-se a limpeza e extração do conteúdo textual, removendo metadados, numerações de cena e marcações temporais, preservando exclusivamente o conteúdo dialógico. Por fim, consolidaram-se os textos em quatro documentos organizados por decênios, formando a base unificada para análise.

A conversão das legendas para texto puro foi um passo essencial no pré-processamento. Realizou-se a limpeza integral dos metadados presentes nos arquivos de legenda, eliminando-se numerações de cena, marcações temporais e elementos de formatação, com preservação exclusiva do conteúdo dialógico original. Implementou-se ainda tratamento adequado para garantir a integridade dos caracteres especiais da língua portuguesa. Após esse processo de purificação textual, procedeu-se à consolidação das legendas por período cronológico, gerando quatro documentos textuais correspondentes às décadas estudadas, que constituíram a base unificada para as análises computacionais subsequentes.

\subsection{Valores Sociais}
A primeira análise focou na identificação e quantificação de valores sociais presentes no corpus. Implementou-se um sistema de categorização baseado em cinco dimensões conceptuais fundamentais, operacionalizadas através de um léxico especializado. O protocolo analítico incorporou a segmentação do corpus em unidades contextuais, permitindo a captura de padrões de recorrência temática. A metodologia empregou contagem contextualizada de unidades lexicais predefinidas, habilitando a mensuração sistemática de construtos sociais complexos ao longo do eixo temporal investigado.

\subsection{Dimensões Emocionais}
A segunda análise dedicou-se ao mapeamento de dimensões emocionais através de uma estrutura baseada em categorias afetivas fundamentais. O protocolo utilizou 46 descritores lexicais validados para identificação de estados emocionais. A metodologia estendeu-se além da contagem lexical simples, incorporando análise de distribuição contextual e densidade emocional. Este enfoque permitiu capturar não apenas a frequência de marcadores emocionais, mas também seus padrões de ocorrência e variação diacrônica no tecido discursivo.

\subsection{Evolução do Vocabulário}
A terceira análise concentrou-se na evolução do vocabulário especializado mediante o monitoramento de 15 unidades lexicais nucleares. A metodologia empregou protocolos de correspondência padronizada com expressões regulares para garantir precisão na identificação de ocorrências semanticamente válidas. O enfoque analítico permitiu a quantificação de padrões de distribuição temporal e trajetórias de uso, assegurando a detecção confiável de transformações no vocabulário especializado ao longo do período estudado.


\section{Resultados e discussão}

\section{Conclusões }


\bibliographystyle{abntex2-alf} 
\bibliography{bibliography}

\end{document}