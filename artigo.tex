\documentclass[twocolumn, 9pt]{article}

\usepackage[utf8]{inputenc}
\usepackage[T1]{fontenc}
\usepackage[brazil]{babel}


\usepackage[left=48pt, right=42pt, top=48pt, bottom=60pt]{geometry}


\usepackage{mathpazo}
\usepackage[scaled]{helvet}


\usepackage[alf, abnt-etal-cite=4, abnt-year-extra=plain]{abntex2cite}

\usepackage{xcolor}
\usepackage[explicit]{titlesec}
\usepackage{lineno}



\definecolor{DarkBlue}{RGB}{0,48,58}

\titleformat{\section}
  {\large\sffamily\bfseries\color{DarkBlue}}
  {\thesection.}
  {0.5em}
  {#1}
  []


\usepackage{graphicx}


\begin{document}


\twocolumn[
  \begin{center} 
    {\sffamily\bfseries\huge Estudo e modelagem para legendas de filmes infantis\par}
    \vspace{1.5em}
    {\large Leoman Cássio Almeida dos Santos\par}
    {\large Marcos Batista Figueredo\par}
    \vspace{1em}
    {\normalsize Universidade do Estado da Bahia\par}
    \vspace{2em}
  \end{center}


  \noindent\textbf{Abstract.} 
  Este estudo apresenta uma análise computacional em larga escala de um corpus de 433 legendas de filmes infantis, 
  abrangendo 4 décadas. Utilizando técnicas de Processamento de Linguagem Natural (PLN), o trabalho foca em quatro eixos
   principais: (1) a transmissão de valores sociais ao decorrer das décadas; (2) a construção linguística dos personagens; (3)
    a evolução do vocabulário e da temática; e (4) a representação e variação das emoções nos diálogos. Para investigar esses 
    eixos, a metodologia emprega classificação supervisionada para a análise de valores, modelos de detecção de emoção e 
    embeddings diacrônicos para capturar a variação das emoções. Os resultados revelarão padrões 
  temporais claros na representação de temas como família, diversidade e tecnologia, além de uma transformação mensurável na complexidade linguística e no perfil emocional dos personagens.
  \par\vspace{1em} 

  \noindent\textbf{Keywords:} Processamento de Linguagem Natural; Legendas; Processamento computacional
  \par\vspace{3em} 
]


\section{Introdução}

O cinema infantil transcende o mero entretenimento para se estabelecer como um dos mais poderosos agentes de socialização da cultura contemporânea. Como artefatos culturais, os filmes destinados a esse público funcionam como espelhos que refletem as ansiedades, aspirações e valores da sociedade em que foram criados, ao mesmo tempo em que atuam como moldes, ensinando às novas gerações sobre normas sociais, ética e a estrutura do mundo. Conforme aponta a literatura da área, o filme é um recurso de fácil acesso que exerce poder de atração e influência sobre as crianças, tornando-se um objeto de estudo relevante para compreender as mensagens transmitidas à infância \cite{Borges2023Filmes}.

\section{Materiais e Métodos}


\section{Resultados Esperados}


\section{Conclusão}


\bibliographystyle{abntex2-alf} 
\bibliography{bibliography}

\end{document}